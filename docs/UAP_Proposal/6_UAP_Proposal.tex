% MIT 6.UAP Proposal
% Author: Michael Mekonnen

\documentclass[12pt]{amsart}

% Packages
\usepackage{hyperref}

\title{6.UAP Proposal: New teaching tools for 6.01}
\author{Michael Mekonnen \\ \textbf{Supervisor:} Dennis Freeman}
\date{February 28, 2013}

\begin{document}

\maketitle

\section{Introduction}

In this paper, I outline what I plan to do for my Undergraduate Advanced Project (6.UAP) assignment. I will be developing the basis and the written proposal for my Masters of Engineering Thesis project. I plan to work with Professor Dennis Freeman and other 6.01 instructors to produce new teaching tools for the course.

\section{6.01: Introduction to EECS I}

6.01 is the first Course 6 class for a large percentage of the students in the department. It is, therefore, a very important course, one that instills key principles of EECS in students new to the field. The class introduces several EECS topics including Object Oriented Programming, Search, Signals and Systems, Circuits, and Probabilistic State Estimation. Students get a first-hand experience learning the material by completing several lab assignments, and in these lab assignments they are exposed to various teaching tools. These teaching tools are a fantastic resource in facilitating the learning process. For instance, Circuits Maximus (CMax) is a program that lets students simulate building circuits on a breadboard. Since its introduction to the course, it has been a great resource for students, especially those  with little or no prior experience building circuits. In my Thesis project, I plan to create two additional teaching tools for 6.01: a Circuit Simulator and a System Simulator.

\section{Circuit Simulator}
The Circuit Simulator is a graphical user interface that lets students enter a schematic drawing of a circuit (containing circuit components relevant to 6.01). This tool can then be used to simulate the resulting outcomes of the circuit. Such a tool can be quite useful for students in completing their lab assignment since it will enable them to test their circuit designs for the desired behaviors before actually building them. In addition to simulating circuits, I plan to have the tool automatically generate the corresponding breadboard layout. Breadboard layout, for circuits with multiple components and connections, is a particularly difficult task, and, hence, this tool will be very useful in letting students spend more of their lab hours on tasks more relevant to learning the course material.

\section{System Simulator}
The System Simulator is another graphical user interface that lets students build and analyze Discrete-Time Linear, Time-Invariant systems. DT LTI system analysis is an important part of 6.01 in which students get a glimpse of Signals and Systems analysis. The tool I plan to build will let students develop an intuition for system analysis through extensive experimentation via a very user-friendly interface.

\section{Why me?}
I have been greatly involved with 6.01 throughout my MIT career. I served as an Lab Assistant for the class twice, and I am currently serving as an Undergraduate Teaching Assistant. Thus, I have a good understanding of students' experiences in the class, and I am motivated to help students enjoy 6.01 as much as I did when I took it. As a Course 6-2 major, I love bridging the small gap between Electrical Engineering and Computer Science whenever possible. With the tools I plan to build for 6.01, I will be able to make extensive use both my EE and CS skills.

\section{Plan of Implementation}
I am currently a senior, graduating in June, 2013. I plan to return to MIT to complete my Masters of Engineering degree in the Fall, 2013 term. To make this a reality, I have already started working on my Thesis project and I have been making significant progress. I have had regular meetings with Professor Freeman to discuss my progress. I will submit my official Thesis proposal well before the end of this term. All my work is publicly available at this GitHub repository: \url{https://github.com/mikemeko/6.01_Tools}.

\end{document}

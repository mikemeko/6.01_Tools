% Chapter for alternative methods to solving the problem.

\chapter{Methods}

In this Section, I discuss my solution to the problem and various alternatives I
considered along the way.

\section{Overview}

I solved this problem by formulating it as a search problem. By this I mean,
given a schematic of a circuit, I start from an empty protoboard, and I consider
the space of all possible protoboard layouts to find the protoboard
corresponding to the schematic at hand. The space of all possible protoboards is
very large \textcolor{red}{\textbf{(?)}}, so I utilize various heuristics to
facilitate the search.

I broke down the problem into two parts. The first task is finding a placement
of all the circuit pieces on the protoboard. The second task is wiring them up
appropriately.

\section{Part 1: Piece Placement}

Let us first consider how to place a set of circuit pieces on the protoboard for
a given circuit schematic.

\subsection{The Pieces}

Any given circuit may contain resistors, Op Amps, pots, motors, head connector
parts, or robot connector parts. For each of these components, we must put down
a corresponding piece on the circuit.

\subsubsection{Resistors}

For the sake of simplicity, and to significantly reduce the search space
\textcolor{red}{\textbf{(?)}}, for every resistor in the schematic, I use one
resistor piece on the protoboard placed in the middle strip of the protoboard
as shown in Figure \ref{fig:piece_placement}. This choice, i.e. allowing the
resistor pieces to only reside in the middle strip of the protoboard, is critical
as the resistor pieces can generally be placed at numerous places on the protoboard.
With this restriction, there are $63$ slots available for one resistor. Without
this restriction, there are a total of $763$ slots available. The restirction is
good when we consider the reduction in the search space size. On the other hand,
the restriction is bad when we
consider the size of circuits the algorithm can layout. Given that the number
of resistors in a typical 6.01 \textcolor{red}{\textbf{(?)}} circuit is very
small, this restriction proves
to be very useful, but we will consider the alternative in Section
\ref{sec:resistors_as_wires}.

\begin{figure}
\begin{center}
\includegraphics{Images/placeholder.jpg}
\caption{Placement of the pieces on the protoboard.}
\end{center}
\label{fig:piece_placement}
\end{figure}

\subsubsection{Op Amps}

Op Amps are the trickiest components to handle because each Op Amp package put
on the protoboard contains two Op Amps within it. Thus, we face the task of
packaging the Op Amps in the schematic in the ``best" possible way, i.e. so as
to require as little work as possible when wiring the pieces together. Equation
\ref{eq:opamp} presents an expression for the value $f(n)$, the number of
different ways
to package together $n$ Op Amps. To get a sense of how many different packagings
are possible, Table \ref{tb:opamp} gives the values of $f(n)$ for various $n$.

\begin{equation}
f(n) = \sum\limits_{k=0}^{\lfloor\frac{n}{2}\rfloor}{\frac{n!}{n!(n - 2k)!}}
\label{eq:opamp}
\end{equation}

\begin{table}
\begin{center}
\begin{singlespace}
\begin{tabular}{c | c}
$n$ & $f(n)$ \\
\hline
\hline
1 & 1 \\
2 & 1 \\
3 & 7 \\
4 & 25 \\
5 & 81 \\
6 & 331 \\
7 & 1303 \\
8 & 5937 \\
9 & 26785 \\
10 & 133651
\end{tabular}
\end{singlespace}
\end{center}
\label{tb:opamp}
\caption{Number of ways of packaging together $n$ Op Amps for various values of
$n$.}
\end{table}

Each Op Amp package is placed in the middle strip of the protoboard, as shown in
Figure \ref{fig:piece_placement}.

\subsubsection{Pots}

Each pot piece can be placed in one of two vertical locations on the protoboard.
Figure \ref{fig:piece_placement} provides an example of both options.

\subsubsection{Head, Motor, and Robot Connectors}

We use a 6-pin connector to connect to a motor, and an 8-pin connector to
connect to either a head or a robot. Each connector can be placed in one of two
vertical locations on the protoboard, as shown in Figure
\ref{fig:piece_placement}.

\subsection{Choosing a Placement}

When choosing a placement of circuit pieces on the protoboard, we have at hand a
plethora of options: each piece can be placed in one of very many places on the
protoboard; each piece has two possible orientations; there are numerous ways of
packaging together Op Amps; etc.

\subsubsection{Simplifications}

I reduce this large number of options by only allowing placements in which no
two pieces share a column. Once again, this is not necessary in general, but the
number of pieces necessary for a typical 6.01 circuit would certainly fit in
this framework.

Next, I specify that there be exactly two columns on the protoboard separating
each consecuitive pair of pieces, unless the pieces are both resistors, in which
case there must be exactly one column separating them. These numbers of columns
were chosen to leave enough space for wiring. Given a set of pieces to be put on
the protoboard, this specification reduces the
problem of choosing a placement for the pieces to finding an \emph{order} of the
pieces together with choosing their respective vertical locations and
orientations.

Given these simplifications, we have various options as to how to pick a
placement.

\subsubsection{Random Placement}

One simple alternative may be to choose a placement randomly. That is, we choose
an Op Amp packaging randomly; we choose an order of the pieces randomly; and we
choose the vertical locations and orientations of the pieces randomly as well.
The advantage of this
approach is that it gives us a placement very quickly without requiring much
computation. On the other hand, we may end up placing two pieces that need to be
connected to each other very far apart, and we will have a difficult time doing
the wiring. Hence, we ought to consider alternatives in which we take into
account the task of wiring. We should try to place the pieces so as to require
as little work during wiring as possible.

\subsubsection{Minimal Heuristic Cost}

The key idea is that if two pieces are meant to be connected together by wires,
then they ought to be placed close to each other on the protoboard. We can
capture this idea by assigning heuristic costs to the placements and choosing
a placement that produces the minimal heuristic cost.

Let us first devise the cost function to achieve this goal. Given a circuit
schematic and a corresponding placement of the circuit pieces on the protoboard,
what do we need to connect with wires? Well, every pair of components in the
schematic that are connected by a wire gives us a corresponding pair of
locations on the protoboard that ought to be connected by wires. However, we can
express this requirement a little bit more concisely. We ought to consider all
of the nodes in the schematic, and find the circuit components in the schematic
that are connected to the respective nodes. Now for each node in the circuit, we
get a set of locations on the protoboard that ought to be interconnected. The
first step in devising the cost function we are looking for is to have a way to
estimate the cost of connecting two locations on the protoboard. A simple such
cost function that comes to mind is the Manhattan distance between the two
locations. Recall that we want to produce aesthetically pleasing protoboard
layouts, and one of the requirements in achieving this goal is only using
horizontal and vertical wires (i.e. no diagonal wires) so the Manhattan distance
cost is appropriate. Given this heuristic cost for connecting two locations with
wires, we can define the heuristic cost for interconnecting the locations
associated with a particular node to be the weight of the minimum spanning tree
of the locations. Now we can define the cost of a placement to be the sum over
all nodes in the circuit of the cost for interconnecting the locations for each
node.

Now that we have a cost function for placements, we can aim to find a
placement with the minimal cost. However, this involves trying all possible
orderings of the pieces with which we are working. For example, if we are trying
to order $10$ pieces, we would need to look at $10! = 3628800$ possible
orderings. Note that this is in addition to searching over all possible ways
of packaging the Op Amps together. It is clear to see that the search for a
minimal cost placement quickly gets out of hand. So we aim to find a placement
that has a very small, though maybe not minimal, cost.

\subsubsection{Small Heuristic Cost}

Algorithm \ref{alg:small_cost_placement} presents a procedure that orders a
given list of pieces in a way that results in a small cost. The algorithm relies
on two ideas. First, once a piece has been placed, all the pieces that are
connected to it will be placed soon after so that it is more likely that those
pieces are placed close to it. Second, we place the pieces with the most nodes
first since those are the one that most likely have connections with many other
pieces.

\begin{algorithm}
\KwData{A list $P$ of circuit pieces.}
\KwResult{A list $R$ of circuit pieces representing a placement.}
\BlankLine
Sort $P$ by number of nodes on the respective pieces\;
$q$ $\leftarrow$ empty Queue\;
$R$ $\leftarrow$ empty List\;
\While{$P$ is not empty}{
Pop the first piece in $P$ and push it onto $Q$\;
\While{$Q$ is not empty}{
$p$ $\leftarrow$ $Q$.pop()\;
Consider all vertical locations and orientations of $p$\;
Place $p$ at an index in $P$ that minimizes the cost of $P$\;
\ForEach{piece $q$ in $P$ connected to $p$}{
Push $q$ onto $Q$\;
}
}
}

\caption{Producing a circuit piece placement with small heuristic cost.}
\label{alg:small_cost_placement}
\end{algorithm}

Using one of the above methods, we can find a placement of circuit pieces on a
protoboard. Our next task is wiring them together to produce a circuit
equivalent to the circuit schematic of interest.

\section{Part 2: Wiring}

\subsection{What do we need to wire together?}

\subsection{Search Infrastructure}

\subsection{Wire all pairs at once}

\subsection{Wire one pair at a time}

\section{Why breakdown problem into two parts?}

\subsection{Treating Resistors as Wires}
\label{sec:resistors_as_wires}

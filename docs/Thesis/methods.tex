% Chapter for alternative methods to solving the problem.

\chapter{Methods}

In this Section, I discuss my solution to the problem and various alternatives I
considered along the way.

\section{Overview}

I solved this problem by formulating it as a search problem. By this I mean,
given a schematic of a circuit, I start from an empty protoboard, and I consider
the space of all possible protoboard layouts to find the protoboard
corresponding to the schematic at hand. The space of all possible protoboards is
very large \textcolor{red}{\textbf{(?)}}, so I utilize various heuristics to
facilitate the search.

I broke down the problem into two parts. The first task is finding a placement
of all the circuit pieces on the protoboard. The second task is wiring them up
appropriately.

\section{Part 1: Piece Placement}

Let us first consider how to place a set of circuit pieces on the protoboard for
a given circuit schematic.

\subsection{The Pieces}

Any given circuit may contain resistors, Op Amps, pots, motors, head connector
parts, or robot connector parts. For each of these components, we must put down
a corresponding piece on the circuit.

\subsubsection{Resistors}

For the sake of simplicity, and to significantly reduce the search space
\textcolor{red}{\textbf{(?)}}, for every resistor in the schematic, I use one
resistor piece on the protoboard placed in the middle strip of the protoboard
as shown in Figure \ref{fig:piece_placement}. This choice, i.e. allowing the
resistor pieces to only reside in the middle strip of the protoboard is critical
as the resistor pieces generally be placed at numerous places on the protoboard.
With this restriction, there are $63$ slots available for one resistor. Without
this restriction, there are a total of $763$ slots available. The restirction is
good when we consider the reduction in the search space size. On the other hand,
the restriction is bad when we
consider the size of circuits the algorithm can layout. Given that the number
of circuits in the typical 6.01 circuit is very small, this restriction proves
to be very useful, but we will consider the alternative in Section
\ref{sec:resistors_as_wires}.

\begin{figure}
\begin{center}
\includegraphics{Images/placeholder.jpg}
\caption{Placement of the pieces on the protoboard.}
\end{center}
\label{fig:piece_placement}
\end{figure}

\subsubsection{Op Amps}

Op Amps are the trickiest components to handle because each Op Amp package put
on the protoboard contains two Op Amps within it. Thus, we face the task of
packaging the Op Amps in the schematic in the ``best" possible way, i.e. so as
to require as little work as possible when wiring the pieces together. Equation
\ref{eq:opamp} presents an expression for the value $f(n)$, the number of
different ways
to package together $n$ Op Amps. To get a sense of how many different packagings
are possible, Table \ref{tb:opamp} gives the values of $f(n)$ for various $n$.

\begin{equation}
f(n) = \sum\limits_{k=0}^{\lfloor\frac{n}{2}\rfloor}{\frac{n!}{n!(n - 2k)!}}
\label{eq:opamp}
\end{equation}

\begin{table}
\begin{center}
\begin{singlespace}
\begin{tabular}{c | c}
$n$ & $f(n)$ \\
\hline
\hline
1 & 1 \\
2 & 1 \\
3 & 7 \\
4 & 25 \\
5 & 81 \\
6 & 331 \\
7 & 1303 \\
8 & 5937 \\
9 & 26785 \\
10 & 133651
\end{tabular}
\end{singlespace}
\end{center}
\label{tb:opamp}
\caption{Number of ways of packaging together $n$ Op Amps for various values of
$n$.}
\end{table}

\subsubsection{Pots}

\subsubsection{Head, Motor, and Robot Connectors}

\subsection{Choosing a Placement}

\subsubsection{Random Placement}

\subsubsection{Minimal Heuristic Cost}

\subsubsection{Small Heuristic Cost}

\section{Part 2: Wiring}

\subsection{What do we need to wire together?}

\subsection{Search Infrastructure}

\subsection{Wire all pairs at once}

\subsection{Wire one pair at a time}

\section{Why breakdown problem into two parts?}

\subsection{Treating Resistors as Wires}
\label{sec:resistors_as_wires}

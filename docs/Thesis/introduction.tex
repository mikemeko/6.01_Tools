%% This is an example first chapter.  You should put chapter/appendix that you
%% write into a separate file, and add a line \include{yourfilename} to
%% main.tex, where `yourfilename.tex' is the name of the chapter/appendix file.
%% You can process specific files by typing their names in at the 
%% \files=
%% prompt when you run the file main.tex through LaTeX.

% Chapter for introduction.

\chapter{Introduction}
\label{ch:intro}

\section{Problem Statement}

In this paper we discuss the problem of automatic protoboard layout generation.
Importantly, we are interested in automatically generating layouts that are
aesthetically pleasing and easy to build. The program that this paper discusses
is strictly geared towards circuits that students would build in the
Introduction to Electrical Engineering and Computer Science I course at MIT (also known as 6.01).

Layout is generally a hard problem for humans. In 6.01, students are tasked with
experimenting with various circuits. We are motivated to solve the layout
problem so as to let students spend most of their time on thinking about how to
design
circuits and almost no time worrying about how to lay it out on a protoboard.
This project aims to build a program that lets students very easily build and
simulate circuit schematics through an intuitive Graphical User Interface (GUI).
After building and testing their circuits with the program, students can then
proceed to building the circuit on a physical protoboard based on the layout
generated by the program. Not only does the program generate a layout, but it
also makes it easy to relate the original schematic drawn by the student to the
generated layout. With this progrem, a student's lab time will be spent mostly
on designing, building, and testing circuits rather than on the much less
instructive task of layout.

\section{Outline}

Chapter \ref{ch:background} describes in detail the terminology that will be
used in this paper and explores the current infrastructure available for
6.01 students as well as previous work done in automatic layout generation.
Chapter \ref{ch:methods} discusses how we solve the problem, including various
alternatives considered in each part of the solution, and how we evaluate
our solution. Chapter \ref{ch:results} presents data to compare the various
alternatives discussed in Chapter \ref{ch:methods} and also evaluates the final
algorithm on a large test dataset. Finally, Chapter \ref{ch:discussion}
presents arguments for the choices made in our solution to the the problem and
also elaborates upon the results presented in Chapter \ref{ch:results}.

% This is an example of how you would use tgrind to include an example
% of source code; it is commented out in this template since the code
% example file does not exist.  To use it, you need to remove the '%' on the
% beginning of the line, and insert your own information in the call.
%
%\tagrind[htbp]{code/pmn.s.tex}{Post Multiply Normalization}{opt:pmn}


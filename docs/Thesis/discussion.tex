% Chapter where we explain the results.

\chapter{Discussion}
\label{ch:discussion}

\section{Explaining the Results}

Chapter \ref{ch:results} presented quantitative data to compare the various
alternatives we have in solving the protoboard layout problem. Here, we will
analyze that data and give reasonings for why we obtained the results
that we obtained.

\subsection{Comparing placement methods}

The results presented in Section \ref{sec:compare_placement} suggest very
interesting behaviors of our two placement methods: the blocking method and the
distance method. In this section, we will discuss several such behaviors and
draw on the appropriate data for support.

In Figure \ref{fig:placement_success} and Table \ref{tb:placement_success} we
see that the distance method exceeds the blocking method in number of circuits
solved $10$ times out of $10$ as well as in the number of circuits solved $0$
to $2$ times out of $10$. On the other hand, the blocking method exceeds the
distance method in number of circuits solved from $3$ to $9$ times out
of $10$. If our primary goal is avoiding failure, then these results suggest
that the blocking method is a better placement method. If, on the other hand,
our primary goal is being consistent on success (while not necessarily being
successful), then these results perhaps suggest that the distance method is
better. In a sense, these results suggest that while the blocking method
generally produces placements that are easier for the wiring stage, it has more
variability in the placements it produces for the same circuit. The distance
method, on the other hand, produces worse placements (i.e. placements that are
harder to wire), but has much less variability. This observation from the data
makes intuitive sense as there are likely to be more placements that are minimal
in blocking rows than placements that minimize pairwise distances between pairs
of protoboard locations that need to be connected.

Despite the important differences presented above, it is interesting to note
that the
success rates of the two placement methods are not that different form each
other. Looking at the percentages given in Table \ref{tb:placement_success}, we
see that the success rates are very comparable.

When we consider success rate as a function of the complexity of circuits, we
reach some interesting patterns. From Figure \ref{fig:placement_success_trend} we
can gather that when the number of pins in the circuit is less than roughly $26$,
the blocking method generally has a higher success rate, but when the number of
pins is greater than $26$, the success rates of the two methods are very
comparable. This suggests that the blocking method is a better placement method
for the less complex circuits, but does not do any better on the more
complex ones. These results compliment what we found above. These results
suggest that the blocking method generally produces placements that are easy to
wire. These placements are easy in that the search in wiring is less likely to
get stuck while trying to connect a pair of locations. This is precisely a result
of the fact that the blocking method attempts to free as many rows as possible.
The blocking method is less effective on the more complex circuits because
despite freeing rows, it may ask for elaborate connections to be made, i.e.
connections between distant pairs of locations on the protoboard. The distance
method, on the other hand, attempts to avoid this problem by trying to minimize
the distances between the locations that need to be connected. Hence we would
not expect the blocking method to be better on the more complex circuits.
Interestingly, it does not seem to be worse on the more complex circuits either.

As we would expect, we certainly observe that as the complexity of the circuits
increases, success rate generally decreases for both of the placement methods.
The curves in Figure \ref{fig:placement_success_trend} seem to shoot back up
at the far end of the figure, but this is a result of the fact that there are
very few circuits at that end of the figure, on which the algorithm happened to
be consistently successful.

Let us now consider wiring time as the basis for comparison. Once again, in
Figure \ref{fig:placement_time_trend} we notice a very interesting separation
at roughly $26$ pins per circuit. When the number of pins in the circuits is
less than $26$, the wiring times for the two methods are very comparable.
When the number of pins in the circuits is greater than $26$, however, the
blocking methods consistently takes longer. This is very much related to the
discussion above that the blocking method may often ask for elaborate connections
to be made. As the complexity of circuits increases, the blocking method will
require the wiring step to make more elaborate connections than would the distance
method. Hence, when we are using the blocking method and the algorithm does succeed,
we would certainly expect the wiring step to take longer than if we had used the
distance method.

As we would expect, we certainly see that as the complexity of the circuits
increases, the amount of time spent by the wiring step also increases. As we did
for the success rate trends as a function of circuit complexity, we observe that
there are outliers at the far end of the figure due to a very small sample of the
most complex circuits in the randomly generated schematic dataset.

Finally, let us look at layout complexity as the basis for comparison. Figure
\ref{fig:placement_quality_trend} presents graphs that compare numbers of wires,
numbers of wire crosses, and total wire lengths. We first observe that the
number of wires used by the two methods are almost identical. As the complexity
of the circuit increases, we see that the blocking method uses more wires than
does the distance method, but by and large the values are very comparable. This
makes intuitive sense as we are rarely required to use more wires than
absolutely necessary (keeping wires horizontal and vertical) to connect a pair
of locations (?). When we look at the number of wire crosses in the layouts, we
see that the blocking method consistently results in more wire crosses. Similarly,
when we look at the total length of wires used in the layouts, the blocking
method exceeds the distance method consistently, with the difference getting
higher as complexity increases. This can be explained by the fact that the
blocking method may require elaborate connections, especially as the circuits
get more complex.

It is difficult to conclusively pick a better placement method from these results.
It is clear that both methods have their strengths and weaknesses. In Section
\ref{sec:method_combination} we will discuss how combining these two methods can
get us the best of both worlds (?).

\subsection{Comparing wiring methods}

\subsection{Comparing resistor treatments}

\subsection{Comparing search methods}

\subsection{Putting them all together}
\label{sec:method_combination}

\section{Remarks}

\textit{Why are these results encouraging? What are their implications? Relate
back to Introduction to Thesis. What could have been done differently?}

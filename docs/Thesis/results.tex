% Chapter in which we explore how well various approaches do.

\chapter{Results}
\label{ch:results}

In this chapter, we compare the various alternatives discussed in the previous
chapter. We test each of the alternatives on the randomly generated dataset
containing \q test schematics. All methods are evaluated on the same dataset for
appropriate comparison.

To assess the goodness of an alternative method, we give:
\begin{enumerate}
\item A bar graph in which each bar represents one run and indicates (1) whether
the alternative was successful (colored green) or unsuccessful (colored red)  on
that test schematic, and (2) the amount of time that run took (height of each
bar). This graph gives a general overview of how well the method performed on
the test dataset.
\item The success rate of the method on the test dataset. The higher this value,
the better the method.
\item The average and standard deviation of success times. The lower the average,
the better the method.
\item The average and standard deviation of failure times. The lower the average,
the better the method.
\item Among the solved schematics, the average and standard deviation of the
number of wires used. The lower the average, the better the method.
\item Among the solved schematics, the average and standard deviation of the
total wire length of the wires on the protoboard. The lower the average, the
better the method.
\item Among the solved schematics, the average and standard deviation of the
number of wire crosses on the protoboard. The lower the average, the better the
method.
\end{enumerate}

To assess the effect of various aspects of schematics on the method, we give
tables in which we consider the values of each of the following items, and
record how success rate, average success time, and average failure time depend on the values:
\begin{enumerate}
\item Number of components.
\item Number of nodes in the circuit.
\item Number of resistors.
\item Number of Op Amps.
\item Number of pots.
\item Number of motors.
\item Number of Head Connectors ($0$ or $1$).
\item Number of Robot Connectors ($0$ or $1$).
\end{enumerate}

Discussion and analysis of the results follow in Chapter \ref{ch:discussion}.

% Chapter for background.

\chapter{Background}

In this section we will discuss essential background information to this project.
First we discuss the specific terminology used in this paper. Next we discuss
previous work done that relates to this project.

\section{Technical Background}

As already mentioned, this project aims to produces a new teaching tool for the
introductory course 6.01. We will now discuss the scope of circuits in 6.01.

\subsection{What are our circuit components?}

The rudimentary circuit components that students work with are resistors,
operational amplifiers (op amps), and potentiometers (pots). Students start out
by building very simple circuits, and then go on to building
more complicated circuits with time. The simplest circuits that students build
aim to control lego motors in a particular way. In constructing these circuits,
students use $6$-pin \textit{motor connectors} to connect their circuits to
lego motors.
The more complicated circuits students build interact with robots that were
built specifically for the purposes of 6.01. One of the 6.01 robots is displayed
in Figure \ref{fig:robot}. Students use $8$-pin \textit{robot connectors} to
connect their circuits to robots. The robots can be equipped with heads that have
vision capabilities. Each head has a rod attached to a potentiometer. Also attached
to the rod are a lego motor and a plate containing two photosensors positioned
at a $90\,^{\circ}$ angle from each other. The
photosensors are used to serve as eyes for the robot. This setup allows us to
turn the head by controlling the motor and inquire the current position of the
head by probing the pot. Figure \ref{fig:robot} displays a robot with a head.
Students use $8$-pin \textit{head connectors} to connect their circuits to
robot heads.

All together, our components are resistors, op amps, pots, motor connectors,
robot connectors, and head connectors.

\begin{figure}
\begin{center}
\includegraphics{Images/placeholder.jpg}
\caption{6.01 robot.}
\label{fig:robot}
\end{center}
\end{figure}

\subsection{What is a circuit schematic?}

Throughout this paper, the term \textit{circuit schematic} will refer to a
drawing or a sketch of a circuit containing its components and all the
interconnections between the components drawn as wires. This is what one would
sketch on a piece of paper in the process of designing a circuit. Figure
\ref{fig:schematic} presents an example of a circuit schematic.

\begin{figure}
\begin{center}
\includegraphics{Images/placeholder.jpg}
\caption{Sample circuit schematic.}
\label{fig:schematic}
\end{center}
\end{figure}

\subsection{What is a protoboard?}

Protoboards are constructs that make it easy to quickly build and test small
circuits. They present a $2$-dimensional array of cleverly interconnected dots
in which circuit pieces and wires can be inserted. Figure
\ref{fig:physical_protoboard} presents an example of a physical protoboard. In
the orientation depicted in Figure \ref{fig:physical_protoboard}, a protoboard
has $4$ groups of rows: the first $2$ rows, the next $5$ rows, the next $5$
rows, and finally the last $2$ rows. In the first and last groups, the dots on
the protoboard are interconnected horizontally. In the middle
two groups, the dots on the protoboard are interconnected vertically. This
interconnection scheme is depicted in Figure \ref{fig:physical_protoboard}.

\begin{figure}
\begin{center}
\includegraphics{Images/placeholder.jpg}
\caption{Physical protoboard.}
\label{fig:physical_protoboard}
\end{center}
\end{figure}

\subsection{What is a protoboard layout of a circuit schematic?}

The protoboard layout of a given schematic is the placement of circuit pieces
and wires on a protoboard that corresponds to the schematic. This is done by
placing the appropriate pieces on the protoboard and then appropriately
interconnecting them with wires as prescribed by the schematic. As an example,
Figure \ref{fig:eg_s_to_pb} presents the protoboard layout corresponding to the
example schematic shown in Figure \ref{fig:schematic}.

\begin{figure}
\begin{center}
\includegraphics{Images/placeholder.jpg}
\caption{Protoboard layout for the schematic in Figure \ref{fig:schematic}.}
\label{fig:eg_s_to_pb}
\end{center}
\end{figure}

For each of the circuit components we are interested in, there is a corresponding
circuit piece that may be inserted into the protoboard. The one exception is that
op amps come in pairs. That is, each op amp circuit piece that is inserted in the
protoboard actually contains two op amp components within it. This raises an
important design question when we layout a schematic: what is the best way to
group together the op amps in a schematic to result in the ``best" layout? In
answering this question, the designer must have some criteria for what makes a
layout ``good." While there are no conclusive answers for this question,
general rules of thumb are (in no particular order):

\begin{itemize}
\item The layout should have no crossing wires.
\item The layout should not have any wires that cross circuit pieces.
\item The layout should only have horizontal and vertical wires.
\item The layout should have as few wires as possible.
\item The total length of wires in a layout should be as small as possible.
\end{itemize}

Given the background information discussed thus far, the goal of our project is
generating a ``good" protoboard layout from circuit schematics automatically.

\section{Previous Work}

\subsection{Current tools in 6.01}

\subsection{Current work in automatic layout}
